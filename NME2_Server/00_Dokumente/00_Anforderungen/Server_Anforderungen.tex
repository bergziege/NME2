\documentclass[11pt,a4paper,notitlepage]{article}
\usepackage[latin1]{inputenc}
\usepackage{amsmath}
\usepackage{amsfonts}
\usepackage{amssymb}
\usepackage{ngerman}
\usepackage[colorlinks=false,pdfborder=0]{hyperref}
\author{Bernd Boruttau}
\title{NME2 Server\\Anforderungen}
\date{18.10.2010}
\begin{document}
\maketitle

\section{Inhalt}
Dieses Dokument beschreibt die Anforderungen an die Oberfl�che bzw. die Funktionen des NME Server im Frontend. Alle Funktionen das Backend (Client / Manager) betreffend wurden absichtlich ausgelassen.

\section{Beschreibung NME2}
Bei NME bzw. NME2 handelt es sich um ein Addon f�r dei FSX Flugsimulation von Mircosoft\copyright.
Das Programm erm�glicht dabei die ferngesteuerte Verwaltung bzw. Platzierung von Objekten im Simulator. Die Objekte werden als Missionen verwaltet und k�nnen im Manager bearbeitet werden.
Neben dem Client (FSX Seitig) und dem Manager (Verwaltung) besteht das Addon aus einer Webanwendung.
�ber diese , Server genannte Anwendung, kann sich der Benutzer einen �berblick �ber die Missionen bzw Objekte und andere Teilnehmer verschaffen.

\section{Funktionalit�ten des Servers}
Der Server soll dem Benutzer folgende Funktionalit�ten bieten:
\begin{itemize}
\item �bersicht �ber alle am System angemeldeten Teilnehmer
\item �bersicht �ber die einzelnen "Missionen" und ihrer Objekte.
\item �bersicht �ber alle auf diesem Server zur Verf�gung stehenden Objekte
\end{itemize}
\paragraph{}
Auf eine Benutzerverwaltung wird zum jetzigen Zeitpunkt bewusst verzichtet!\\
Es sind daher keine Login oder Profilbereiche vorzusehen.
\paragraph{}
Folgend eine n�here Beschreibung der einzelnen Punkte und Details.

\subsection{Teilnehmer�bersicht}
\paragraph{}
Alle Teilnehmer sollen auf einer Karte dargestellt werden. Zus�tzlich sollen zu einem einzelnen, frei w�hlbaren Teilnehmer Details eingeblendet werden k�nnen. Die Teilnehmer sollen zus�tzlich zur Kartenansicht tabellarisch angezeigt werden. W�nschenswert w�re eine "Echtzeit"-Anzeige der Teilnehmer.

\paragraph{}
Als Details zu einen Teilnehmer z�hlen z.b. seine Koordinaten, H�he, Geschwindigkeit und die Missionen die f�r diesen Teilnehmer aktiv sind.
Des weiteren soll eine direkte Verlinkung der Missionen des Teilnehmers zu den Missionsdetails erfolgen.

\subsection{Missions�bersicht}
\paragraph{}
Zus�tzlich zu den Teilnehmern sollen die Missionen auf der Karte kenntlich gemacht werden (Mittenposition und Radius). Au�erdem sollen die einzelnen Elemente einer oder mehrerer Missionen angezeigt werden k�nnen.
 \paragraph{}
Zus�tzlich zur Kartenansicht soll eine tabellarische Auflistung der Missionen erfolgen. Ausgehend von der Tabelle sollen die Missionsdetails in Textform angezeigt werden k�nnen.
\paragraph{}
Des weiteren soll es eine Exportm�glichkeit der aktuell auf der Karte angezeigten Daten in das KML bzw. KMZ Format geben.

\subsection{Objekt�bersicht}
\paragraph{}
In dieser Zusammenfassung sollen lediglich alle auf dem Server verf�gbaren Objekte angezeigt werden. Es ist auf eine sinnvolle Anzahl von Elementen Pro Seite zu achten (maximal 50 Elemente).\\
Die Elemente sollen dabei mit ihren hinterlegten Daten sowie einem Bild dargestellt werden.
\end{document}